\documentclass[a4paper]{article} 
\input{head}


\begin{document}

%-------------------------------
%	TITLE SECTION
%-------------------------------

\fancyhead[C]{}
\hrule\bigskip % Upper rule
\begin{minipage}{0.295\textwidth} 
\raggedright
\normalsize
KEXUAN SONG \hfill\\   
40119435\hfill\\
\end{minipage}
\begin{minipage}{0.4\textwidth} 
\centering 
\Large
\textbf{Vision Document}\hfill\\
\textbf{E-Concordia drive}\hfill\\ 
\normalsize 
\end{minipage}
\begin{minipage}{0.295\textwidth} 
\raggedleft
\today\hfill\\
\end{minipage}
\medskip\hrule 
\bigskip

%-------------------------------
%	CONTENTS
%-------------------------------

\section{Introduction}
The purpose of this document is to analyze and synthesize all the information about the situation of online learning, in order to generate comprehensive overview of the E-learning drive platform with needs and possible features.  It focuses on the users’ and stakeholders’ needs, the using environments and the reason of those needs. It also provides the proposal of the way that E-learning platform meets these needs.
\subsection{References}


\section{Positioning}
\subsection{Problem Statement}
\newcolumntype{a}{ >{\columncolor{gray}} c}
\begin{table}[htb]
\begin{tabular}{|a|p{11cm}|}
\hline
The problem of                 & Lacking platform to learn and practice the driving lessons online\medskip \\ \hline
Affects                        & Student, trainer, driving school \\ \hline
The impact of which is         & That students are not able to learn and practice at home, they can only spend more time travel to the driving school to learn on a specific time. Trainers has to repeat themselves many time to teach different group of students. The driving school has to spend extra time to make arrangement of class schedule and personnel. \\ \hline
A successful solution would be & Provide classes online so the students can access the class anytime and anywhere, and provide a platform that student, trainer and driving school can both access to coordinate. \\
\hline
\end{tabular}
\end{table}
\bigskip
\subsection{Product Position Statement}
\begin{table}[htb]
\begin{tabular}{|a|p{12.2cm}|}
\hline
For                 & Driving schools\medskip \\ \hline
Who                        & Needs to move their resources online \\ \hline
The E-Concordia drive         & is a software product \\ \hline
That & trainers can upload the study resources under supervision and students can study and practice online\\ \hline
Unlike & The current state that students have to be at classroom to study and practice and the trainers have to teach the same content multiple times.  \\ \hline
Our product & Makes the learning resources accessible everywhere and allows the reduce the workload of trainers. \\ 
\hline
\end{tabular}
\end{table}
\bigskip

\section{Stakeholder Descriptions}
\subsection{Stakeholder Summary}
\pagebreak
\begin{table}[ht]
\begin{tabular}{|l|p{6cm}|p{6cm}|}
\hline
Name & Description & Responsibilities \\ \hline
Project manager & The head of the development team & Manages the whole development process of development, makes sure the team meets the schedule, budget and requirement. \\ \hline
Development team & The people who perform the work of the project. & Develop the project that customer needs under the project manager’s lead. \\ \hline
Driving school & The end customer who pays for the product & Provides funding of development. \\ \hline
Software company & The company who takes responsibility of development & Provides resources and personnel for the development. \\
\hline

\end{tabular}
\end{table}

\subsection{User Summary}

\begin{table}[h]
\centering
\begin{tabular}{|l|p{5cm}|p{5cm}|l|}
\hline
Name & Description & Responsibilities & Stakeholder \\ \hline
Student & The end user who registered to the driving school to take courses & takes the online courses & Self-represented \\ \hline
Trainer & The end user who uploads the course resources and monitor the students’ study & Upload course resources and make study plans & Self-represented \\ \hline
Admin & The end user who performs quality check of the course content & edits, comments, approves and publishes  lessons uploaded by a trainer & Self-represented \\ 
\hline

\end{tabular}
\end{table}

\bigskip
\subsection{User Environment}



\bigskip
\subsection{Key Stakeholder or User Needs}



\bigskip
\section{Product Overview}
\subsection{Product Perspective}


\bigskip
\subsection{Assumptions and Dependencies}



\bigskip
\section{Product Features}
\subsection{Log in}
Trainer and admin shall provide valid username and password to log in to the E-concordia drive. 
Students shall provide the student number, traffic number and file number to log in.
\bigskip
\subsection{System Update}
The system shall enable admin to view the updates of the system with details, including updated lessons, updated comments and new quality comment. The system shall protect the information from user other than admin.
\bigskip
\subsection{Comment}
Admin shall make comments to slides uploaded by trainers and track the correction status of the comments.  Trainer would get notification if their lessons were commented. The status would update once trainer fix the issue.
\bigskip
\subsection{Lesson Management}
Admin shall view, create, edit, and delete lessons. Admin shall view and edit all the information of lessons including slides and quiz details. Admin can also create and edit slide and quiz for a specific lesson. A lesson will only be published after admin approves. Admin shall also create a new lesson with lesson name, description, language and icon specified.
Trainer shall view and edit a lesson (only for a new version). Trainer can only delete lesson draft, modify and delete quiz before the lesson is published.
\bigskip
\subsection{Quiz}
The system provides many different types of quiz: True/False, correct answer select, drag\&drop match, and reorder arrange. The quiz is a part of a lesson, and it would give real-time feedback. The quiz can be created by admin or trainer. Trainer can only edit quiz when the course is not published.
\bigskip
\subsection{Learning}
The system display lessons to student in a specific order and student must attend lessons in sequence. The system also tracks the progress of course and shows the percentage of completion. A lesson is considered finished when the slides and quiz are both completed. Student shall repeat lessons already finished. After the expiry date of training, the student shall not access the course material. 
\bigskip
\subsection{E-learning slider}
E-learning slider displays the content slides. It has interactive screen and transcript feature. It also allows student to skip the attended slides. If student logs out, the slider will resume from the slide student left. Once all slides are viewed by student, the system proceeds to quiz session. 
\bigskip


\section{Other Product Features}
Applicable standards. The system must comply with existing web standards(HTTP, HTML DOM,etc).
Performance Requirements:
The system must be able to handle 50000 users at the same time with 95\% of response lower than 9 seconds. 
The system response time must be lower than 10 seconds.
Documentation Requirements: 
No user manual. The system must be easy enough for a user to use without manual.
Online help. A user guide need to be provided when a user log in for the first time.

\end{document}
