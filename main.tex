\documentclass[a4paper]{article} 
\addtolength{\hoffset}{-2.25cm}
\addtolength{\textwidth}{4.5cm}
\addtolength{\voffset}{-3.25cm}
\addtolength{\textheight}{5cm}
\setlength{\parskip}{0pt}
\setlength{\parindent}{0in}

%----------------------------------------------------------------------------------------
%	PACKAGES AND OTHER DOCUMENT CONFIGURATIONS
%----------------------------------------------------------------------------------------

\usepackage{blindtext} % Package to generate dummy text
\usepackage{charter} % Use the Charter font
\usepackage[utf8]{inputenc} % Use UTF-8 encoding
\usepackage{microtype} % Slightly tweak font spacing for aesthetics
\usepackage[english, ngerman]{babel} % Language hyphenation and typographical rules
\usepackage{amsthm, amsmath, amssymb} % Mathematical typesetting
\usepackage{float} % Improved interface for floating objects
\usepackage[final, colorlinks = true, 
            linkcolor = black, 
            citecolor = black]{hyperref} % For hyperlinks in the PDF
\usepackage{graphicx, multicol} % Enhanced support for graphics
\usepackage{xcolor} % Driver-independent color extensions
\usepackage{marvosym, wasysym} % More symbols
\usepackage{rotating} % Rotation tools
\usepackage{censor} % Facilities for controlling restricted text
\usepackage{listings, style/lstlisting} % Environment for non-formatted code, !uses style file!
\usepackage{pseudocode} % Environment for specifying algorithms in a natural way
\usepackage{style/avm} % Environment for f-structures, !uses style file!
\usepackage{booktabs} % Enhances quality of tables
\usepackage{tikz-qtree} % Easy tree drawing tool
\tikzset{every tree node/.style={align=center,anchor=north},
         level distance=2cm} % Configuration for q-trees
\usepackage{style/btree} % Configuration for b-trees and b+-trees, !uses style file!
\usepackage[backend=biber,style=numeric,
            sorting=nyt]{biblatex} % Complete reimplementation of bibliographic facilities
\addbibresource{ecl.bib}
\usepackage{csquotes} % Context sensitive quotation facilities
\usepackage[yyyymmdd]{datetime} % Uses YEAR-MONTH-DAY format for dates
\renewcommand{\dateseparator}{-} % Sets dateseparator to '-'
\usepackage{fancyhdr} % Headers and footers
\pagestyle{fancy} % All pages have headers and footers
\fancyhead{}\renewcommand{\headrulewidth}{0pt} % Blank out the default header
\fancyfoot[L]{} % Custom footer text
\fancyfoot[C]{} % Custom footer text
\fancyfoot[R]{\thepage} % Custom footer text
\newcommand{\note}[1]{\marginpar{\scriptsize \textcolor{red}{#1}}} % Enables comments in red on margin

%----------------------------------------------------------------------------------------


\begin{document}

%-------------------------------
%	TITLE SECTION
%-------------------------------

\fancyhead[C]{}
\hrule\bigskip % Upper rule
\begin{minipage}{0.295\textwidth} 
\raggedright
\normalsize
KEXUAN SONG \hfill\\   
40119435\hfill\\
\end{minipage}
\begin{minipage}{0.4\textwidth} 
\centering 
\Large
\textbf{Vision Document}\hfill\\
\textbf{E-Concordia drive}\hfill\\ 
\normalsize 
\end{minipage}
\begin{minipage}{0.295\textwidth} 
\raggedleft
\today\hfill\\
\end{minipage}
\medskip\hrule 
\bigskip

%-------------------------------
%	CONTENTS
%-------------------------------

\section{Introduction}
The purpose of this document is to analyze and synthesize all the information about the situation of online learning, in order to generate comprehensive overview of the E-learning drive platform with needs and possible features.  It focuses on the users’ and stakeholders’ needs, the using environments and the reason of those needs. It also provides the proposal of the way that E-learning platform meets these needs.
\subsection{References}


\section{Positioning}
\subsection{Problem Statement}
\newcolumntype{a}{ >{\columncolor{gray}} c }
\begin{table}[htb]
\begin{tabular}{|a|p{11cm}|}
\hline
The problem of                 & Lacking platform to learn and practice the driving lessons online\medskip \\ \hline
Affects                        & Student, trainer, driving school \\ \hline
The impact of which is         & That students are not able to learn and practice at home, they can only spend more time travel to the driving school to learn on a specific time. Trainers has to repeat themselves many time to teach different group of students. The driving school has to spend extra time to make arrangement of class schedule and personnel. \\ \hline
A successful solution would be & Provide classes online so the students can access the class anytime and anywhere, and provide a platform that student, trainer and driving school can both access to coordinate. \\
\hline
\end{tabular}
\end{table}


\bigskip

%------------------------------------------------

% \section{Second Exercise}
% \blindtext
% \subsection{First Subtask}

% \bigskip

%------------------------------------------------

\end{document}
