\documentclass[a4paper]{article} 
\input{head}

\begin{document}

%-------------------------------
%	TITLE SECTION
%-------------------------------

\fancyhead[C]{}
\hrule\bigskip % Upper rule
\begin{minipage}{0.295\textwidth} 
\raggedright
\normalsize
KEXUAN SONG \hfill\\   
40119435\hfill\\
\end{minipage}
\begin{minipage}{0.4\textwidth} 
\centering 
\Large
\textbf{Vision Document}\hfill\\
\textbf{E-Concordia drive}\hfill\\ 
\normalsize 
\end{minipage}
\begin{minipage}{0.295\textwidth} 
\raggedleft
\today\hfill\\
\end{minipage}
\medskip\hrule 
\bigskip

%-------------------------------
%	CONTENTS
%-------------------------------

\section{Introduction}
The purpose of this document is to analyze and synthesize all the information about the situation of online learning, in order to generate comprehensive overview of the E-learning drive platform with needs and possible features.  It focuses on the users’ and stakeholders’ needs, the using environments and the reason of those needs. It also provides the proposal of the way that E-learning platform meets these needs.
\subsection{References}


\section{Positioning}
\subsection{Problem Statement}
\newcolumntype{a}{ >{\columncolor{gray}} c }
\begin{table}[htb]
\begin{tabular}{|a|p{11cm}|}
\hline
The problem of                 & Lacking platform to learn and practice the driving lessons online\medskip \\ \hline
Affects                        & Student, trainer, driving school \\ \hline
The impact of which is         & That students are not able to learn and practice at home, they can only spend more time travel to the driving school to learn on a specific time. Trainers has to repeat themselves many time to teach different group of students. The driving school has to spend extra time to make arrangement of class schedule and personnel. \\ \hline
A successful solution would be & Provide classes online so the students can access the class anytime and anywhere, and provide a platform that student, trainer and driving school can both access to coordinate. \\
\hline
\end{tabular}
\end{table}


\bigskip

%------------------------------------------------

% \section{Second Exercise}
% \blindtext
% \subsection{First Subtask}

% \bigskip

%------------------------------------------------

\end{document}
